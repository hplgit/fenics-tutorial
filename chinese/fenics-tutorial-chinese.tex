\documentclass{tstextbook}

\begin{document}

\tsbook{在Python中解决PDE \\
 FEniCS教程}
       {Hans Petter Langtangen \& Anders Logg}
       {}
       {2017}
       {xxxxx}{xxx--xx--xxxx--xx--x}{0.0}
       {Chinese translation by FNK InfoTech}
       {Wuxi, China}

\input{newcommands}

%---------------------------------------------------------------------------
% Chapters
%---------------------------------------------------------------------------

%---------------------------------------------------------------------------
\chapter{基本原理:解决泊松方程}

\bigskip

\begin{summary}
本章的目标是展示泊松方程如何
所有 PDE 中最基本的都可以用几行快速解决
的 FEniCS 代码。 我们介绍最多
基本 FEniCS 对象:
\texttt{Mesh}, \texttt{Function}, \texttt{FunctionSpace}, \texttt{TrialFunction}, \texttt{TestFunction}。
了解如何编写基本的 PDE 求解器,
包括如何制定数学变分问题,
应用边界条件,调用 FEniCS 求解器和绘图
解决方案。
\end{summary}


\section{数学问题的制定}

许多关于编程语言的书籍都以HelloWorld开头
程序。 读者好奇知道基本的任务是什么
用语言表达,并将文字打印到屏幕上即可
这样的一个任务。 在有限元方法的世界,PDE,
最基本的任务是解决泊松方程。 我们的
因此对应于古典HelloWorld程序
解决以下边界值问题:

\begin{alignat}{2}
- \nabla^2 u(\x) &= f(\x),\quad &&\x\mbox{ in } \Omega,
\label{ftut:poisson1}\\
u(\x) &= \ub(\x),\quad &&\x\mbox{ on } \partial \Omega\tp \label{ch:poisson0:bc}
\end{alignat}

这里,$u = u(\x)$是未知函数,$f = f(\x)$是
规定的函数,$\nabla^2$是Laplace算子
(通常写为$\Delta$),$\Omega$是空间域,而
$\partial \Omega$是$\Omega$的边界。 Poisson问题,
包括PDE $-\nabla^2u = f$和边界条件
$u = \ub$ on $\partial \Omega$,是边界值的一个例子
问题,必须在之前精确地陈述
使用FEniCS开始解决它是有意义的。

在具有坐标$x$和$y$的两个空间维度中,我们可以写出来
Poisson方程为

\begin{equation}
- {\partial^2 u\over\partial x^2} -
{\partial^2 u\over\partial y^2} = f(x,y)\tp
\end{equation}

未知的$u$现在是两个变量$u = u(x,y)$的函数
在二维域$\Omega$。

Poisson方程出现在许多物理环境中,包括
热传导,静电,物质扩散,扭转
弹性棒,非粘性流体流和水波。 而且,
方程出现在数值分割策略中更为复杂
PDE系统,特别是Navier-Stokes方程。

解决边界值问题,如Poisson方程
FEniCS包括以下步骤:

\begin{enumerate}
\item 识别计算域($\Omega$),PDE,它 边界条件和源项($f$)。
\item 将PDE重新定义为有限元变分问题。
\item 编写一个定义计算域的Python程序,变分问题,边界条件和来源
  条款,使用相应的FEniCS抽象。
\item 调用FEniCS来解决边界值问题,并且可选地,扩展程序
计算衍生量,如通量和平均值,以及 可视化结果。
\end{enumerate}

\noindent
现在我们将详细介绍第2--4步。 的主要特点
FEniCS是步骤3和4导致相当短的代码,而a
大多数其他PDE软件框架中的类似程序都需要
更多的代码和技术上困难的编程。

\begin{notice}[什么使FEniCS有吸引力?]
虽然许多软件框架都非常优雅
HelloWorld 的例子
Poisson方程式,FEniCS是我们知道的唯一框架
代码保持紧凑和美观,非常接近数学
即使是数学和算法的复杂性
从笔记本电脑转移到高性能时会增加
计算服务器(集群)。
\end{notice}

\subsection{有限元变分法}
FEniCS是基于有限元法,它是一般的和
高效数学机械的数值解
PDE。 有限元方法的出发点是PDE
以变体形式表达。 不熟悉的读者
变数问题将会简要介绍一下这个话题
在本教程中,但阅读有限元上的正确书
鼓励方法。 经验表明,您可以使用
FEniCS作为解决PDE的工具,即使没有深入的了解
有限元法,只要你有人帮你
将PDE作为变分问题。

将PDE转化为变分问题的基本方法是
将PDE乘以函数$v$,整合得到的等式
通过域$\Omega$,并按部分条款执行整合
与二阶导数。 函数$v$乘以
PDE称为测试功能。 未知函数$u$为
近似被称为试验函数。 试用版和
测试功能也用于FEniCS程序。 试用和测试
功能属于指定的所谓功能空间
功能的属性。

在这种情况下,我们先乘以Poisson方程
通过测试函数$v$并集成$\Omega$:

\begin{equation}
\label{ch:poisson0:multbyv}
-\int_\Omega (\nabla^2 u)v \dx = \int_\Omega fv \dx\tp
\end{equation}
我们这里让$\dx$表示用于集成的差分元素
域$\Omega$。 我们稍后会让$\ds$表示差分
在$\Omega$的边界上整合的元素。

当我们得出变分公式时,一个常见的规则就是我们尝试
保持$u$和$v$的衍生工具的顺序尽可能小
可能。 在这里,我们有一个$u$的二阶空间导数,
这可以转换为$u$和$v$的一阶导数
应用零件整合技术。 公式
读

\begin{equation}
\label{ch:poisson0:eqbyparts}
 -\int_\Omega (\nabla^2 u)v \dx
= \int_\Omega\nabla u\cdot\nabla v \dx - \int_{\partial\Omega}{\partial u\over
\partial n}v \ds,
\end{equation}

其中$\frac{\partial u}{\partial n} = \nabla u \cdot n$是
$u$的衍生物向外向正方向$n$
边界。

变分配方的另一个特点是
测试函数$v$需要在部分消失
解决方案$u$的边界已知。
在现在
问题,这意味着$v = 0$在整个边界$\partial \Omega$。
第二个术语在右边
(\ref{ch:poisson0:eqbyparts})因此消失。 从
(\ref{ch:poisson0:multbyv})和(\ref{ch:poisson0:eqbyparts})
遵循

\begin{equation}
\int_\Omega\nabla u\cdot\nabla v \dx = \int_\Omega fv \dx\tp
\label{ch:poisson0:weak1}
\end{equation}

如果我们要求该方程适用于所有测试函数$v$
一些合适的空间$\hat V$,所谓的测试空间,我们获得一个
明确的数学问题,唯一地决定了
解决方案$u$在于一些可能不同的功能空间
$V$,所谓的试验空间。 我们参考
(\ref{ch:poisson0:weak1})作为弱形式或变体形式
原边界值问题
(\ref{ftut:poisson1})--(\ref{ch:poisson0:bc})。

适当的陈述
我们的变分问题现在如下:
找到$u \in V$这样

\begin{equation} \label{ch:poisson0:var}
  \int_{\Omega} \nabla u \cdot \nabla v \dx =
  \int_{\Omega} fv \dx
  \quad \forall v \in \hat{V}\tp
\end{equation}

现在试用和测试空间$V$和$\hat V$
问题定义为

\begin{align*}
     V      &= \{v \in H^1(\Omega) : v = \ub \mbox{ on } \partial\Omega\}, \\
    \hat{V} &= \{v \in H^1(\Omega) : v = 0 \mbox{ on } \partial\Omega\}\tp
\end{align*}
简而言之,$H^1(\Omega)$是数学上众所周知的Sobolev空间
包含函数$v$,使$v^2$和$|\nabla v|^2$有
有限积分超过$\Omega$(基本上意味着函数
是连续的)。 底层PDE的解决方案必须在于
函数空间中的衍生物也是连续的,但是
Sobolev空间$H^1(\Omega)$允许不连续的函数
衍生物。 $u$的连续性要求较弱
变量语句(\ref{ch:poisson0:var}),作为结果
整合部分,具有很大的实际后果
构造有限元函数空间。 特别是它
允许使用分段多项式函数空间; 即功能
通过简单地将多项式函数拼接在一起构成的空间
域,如间隔,三角形或四面体。

变量问题(\ref{ch:poisson0:var})是连续的
问题:它定义了无穷维的解$u$
功能空间$V$。 Poisson方程的有限元法
找出变分问题的近似解
(\ref{ch:poisson0:var})替换无限维函数
空间$V$和$\hat{V}$通过离散(有限维)试验
测试空间$V_h\subset{V}$和$\hat{V}_h\subset\hat{V}$。 离散变分问题如下:
找到$u_h \in V_h \subset V$这样

\begin{equation} \label{ch:poisson0:vard}
  \int_{\Omega} \nabla u_h \cdot \nabla v \dx =
  \int_{\Omega} fv \dx
  \quad \forall v \in \hat{V}_h \subset \hat{V}\tp
\end{equation}

这个变分问题,连同一个合适的定义
函数空间$V_h$和$\hat{V}_h$,唯一地定义我们的近似值
Poisson方程的数值解(\ref{ftut:poisson1})。 注意
边界条件被编码为试验和测试的一部分
空间。 起初,数学框架可能看起来很复杂
一瞥,但好消息是有限元变分
问题(\ref{ch:poisson0:vard})看起来和连续的一样
变分问题(\ref{ch:poisson0:var})和FEniCS可以
自动解决变量问题,如(\ref{ch:poisson0:vard})!

\begin{notice}[我们的意思是符号$u$和$V$]
关于变数问题的数学文献写了$u_h$
解的离散问题和$u$的解决方案
连续问题 获得(几乎)一对一的关系
在一个问题的数学表达与之间
相应的FEniCS程序,我们将下拉$_h$和使用
$u$用于解决离散问题。
我们将使用$\uex$来确定
解决连续问题,如果我们需要明确区分
两者之间。 类似地,我们将让$V$表示离散有限
元素功能空间,我们寻求我们的解决方案。
\end{notice}

\subsection{抽象有限元变分公式}
\label{ch:poisson0:abstrat}
\index{abstract variational formulation}

原来是方便的介绍下列规范
变量问题的符号:找到$u \in V$这样

\begin{equation}
a(u, v) = L(v) \quad \forall v \in \hat{V}.
\end{equation}

对于Poisson方程,我们有:

\begin{align}
a(u, v) &= \int_{\Omega} \nabla u \cdot \nabla v \dx,
\label{ch:poisson0:vard:a}\\
L(v) &= \int_{\Omega} fv \dx\tp  \label{ch:poisson0:vard:L}
\end{align}


%---------------------------------------------------------------------------
% Bibliography
%---------------------------------------------------------------------------

%\addcontentsline{toc}{chapter}{\textcolor{tssteelblue}{Literature}}
%\printbibliography{}

%---------------------------------------------------------------------------
% Index
%---------------------------------------------------------------------------

\printindex

\end{document}
