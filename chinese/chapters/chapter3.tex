\chapter{有限元求解器画廊}
\label{ch:gallery}

\begin{quote}
本章的目标是展示一系列重要的内容
科学与工程学院的PDEs可以很快得到解决
FEniCS代码行。 我们从热方程开始,继续
用非线性Poisson方程,线性方程
弹性,Navier-Stokes方程式,最后看看如何
解决非线性对流 - 扩散反应的系统
方程。 这些问题说明了如何解决时间依赖性问题
问题,非线性问题,向量值问题和系统
PDEs。 对于每个问题,我们得出变分公式
在Python中以非常类似的方式表达问题
数学。
\end{quote}

\section{热方程式}
\label{ch:fundamentals:diffusion}

\index{heat equation}
\index{time-dependent problem}

作为前一章的Poisson问题的第一个扩展,
我们考虑时间依赖热方程,或时间依赖性
扩散方程。 这是Poisson的自然延伸
描述体内热量的固定分布的方程式
时间依赖的问题。

我们会看到,通过将时间离散成小时间间隔
采用标准时间步法,可以解决热量
通过求解一系列变分问题的方程式,很像
一个我们遇到的Poisson方程。

\subsection{PDE问题}

我们的时间依赖型PDE的模型问题读取

\begin{alignat}{2}
{\partial u\over\partial t} &= \nabla^2 u + f \quad &&\hbox{in }\Omega\times(0, T],
\label{ch:diffusion0:pde1}\\
u &= \ub &&\hbox{on } \partial \Omega\times(0, T],
\label{ch:diffusion0:pde1:bc}\\
u &= \uI &&\mbox{at } t=0\tp
\label{ch:diffusion0:pde1:ic}
\end{alignat}
在这里,$u$随空间和时间而变化,例如,如果空间,则$u=u(x,y,t)$
域$\Omega$是二维的。 源函数$f$和
边界值$\ub$也可能因空间和时间而异。
初始条件$\uI$仅是空间的函数。

\subsection{变化公式}
\label{ftut:timedep:diffusion1}

一个简单的方法来解决时间依赖的PDEs
有限元方法是首先离散时间导数a
有限差分近似,产生一个序列
固定问题,然后将每个静止问题转化为
变分公式

让上标$n$表示时间$t_n$的数量,其中$n$是一个
整数计数时间级别。 例如,$u^n$表示$u$在时间
级$n$。 时间上的有限差分离散化包括
在某个时间级别对PDE进行采样,说$t_{n + 1}$:

\index{time step}

\begin{equation}
\left({\partial u \over\partial t}\right)^{n+1} = \nabla^2 u^{n+1} + f^{n+1}\tp
\label{ch:diffusion0:pde1:tk}
\end{equation}
时间导数可以通过差商近似。
为了简单和稳定的原因,我们选择一个
简单向后的差异:

\index{implicit Euler}
\index{backward difference}

\begin{equation}
\left({\partial u\over\partial t}\right)^{n+1}\approx {{u^{n+1} - u^n}\over{\dt}},
\label{ch:diffusion0:BE}
\end{equation}
其中$\dt$是时间离散参数。
(\ref{ch:diffusion0:pde1:tk})插入(\ref{ch:diffusion0:BE})产生

\begin{equation}
{{u^{n+1} - u^n}\over{\dt}} = \nabla^2 u^{n+1} + f^{n+1}\tp
\label{ch:diffusion0:pde1:BE}
\end{equation}
这是我们的时间离散版本的热方程
(\ref{ch:diffusion0:pde1}),所谓的\emph{backward Euler}或\emph{implicit
  Euler}离散化。

我们可以重新排序(\ref{ch:diffusion0:pde1:BE})
左侧包含未知$u^{n+1}$的条款
右侧仅包含计算的条件。 结果
是假设为$u^{n+1}$的空间(静态)问题的序列
$u^n$从以前的时间步长是已知的:

\begin{align}
u^0 &= \uI, \label{ch:diffusion0:pde1:u0}\\
u^{n+1} - {\dt}\nabla^2 u^{n+1} &=  u^n + {\dt} f^{n+1},\quad n=0,1,2,\ldots
\label{ch:diffusion0:pde1:uk}
\end{align}

给定$\uI$,我们可以解决$u^0$,$u^1$,$u^2$等。

(\ref{ch:diffusion0:pde1:uk})的替代方法,可以是
方便实施,是收集
平等标志一方的所有术语:

\begin{equation}
u^{n+1} - {\dt}\nabla^2 u^{n+1} -  u^{n} - {\dt} f^{n+1} = 0,\quad n=0,1,2,\ldots
\label{ch:diffusion0:pde1:uk2}
\end{equation}

我们使用有限元方法来解决
(\ref{ch:diffusion0:pde1:u0})和任一方程式
(\ref{ch:diffusion0:pde1:uk})或(\ref{ch:diffusion0:pde1:uk2})。 这个
需要将方程式变为弱形式。 像往常一样,我们倍增
通过一个测试函数$v\in \hat V$并将二阶导数合并
部分。 在$u^{n+1}$(这是自然的)中引入符号$u$
程序),由此产生的弱势形式
配方(\ref{ch:diffusion0:pde1:uk})
可以方便地写入
标准符号:

\[ a(u,v)=L_{n+1}(v),\]
哪里

\begin{align}
a(u,v) &= \int_\Omega\left(uv + {\dt}
\nabla u\cdot \nabla v\right) \dx, \label{ch:diffusion0:pde1:a}\\
L_{n+1}(v) &= \int_\Omega \left(u^n + {\dt}  f^{n+1}\right)v \dx\tp
\label{ch:diffusion0:pde1:L}
\end{align}
替代形式(\ref{ch:diffusion0:pde1:uk2})有一个
抽象配方

\[ F_{n+1}(u;v) = 0,\]
哪里

\begin{equation}
F_{n+1}(u; v) = \int_\Omega \left(uv + {\dt}
\nabla u\cdot \nabla v -
(u^n + {\dt} f^{n+1})v\right) \dx\tp
\label{ch:diffusion0:pde1:F}
\end{equation}

除了每个时间步长要解决的变化问题外,
我们还需要近似初始条件
(\ref{ch:diffusion0:pde1:u0})。 这个方程也可以变成a
变分问题:

\[ a_0(u,v)=L_0(v),\]
哪里

\begin{align}
a_0(u,v) &= \int_\Omega uv \dx, \label{ch:diffusion0:pde1:a0}\\
L_0(v) &= \int_\Omega \uI v \dx\tp \label{ch:diffusion0:pde1:L0}
\end{align}
